\documentclass{homework}
\usepackage{bussproofs}
\EnableBpAbbreviations
\usepackage{ctex, hyperref}
\hypersetup{hidelinks,
	colorlinks=true,
	allcolors=blue,
	pdfstartview=Fit,
	breaklinks=true
}

\name{熊泽恩} % Replace (Student Name) with your name.
\id{2022011223}
\term{2024 Spring}
\course{Introduction to Theoretical Computer Science}
\hwnum{6}

%\hwname{(Name)}          % Uncomment and replace (Name) with the type of the
                          % homework (e.g, Assignment, Problem Set, etc.) if you
                          % don't want the document to be labeled as "Homework."
%\problemname{(Name)}     % Uncomment and replace (Name) with the desired label
                          % for problems created with the problem environment.
%\solutionname{(Name)}    % Uncomment and replace (Name) with the desired label
                          % for solutions created with the solution environment.

% Load any other packages you need here.

\begin{document}

\begin{problem}
  Prove that if $\P = \NP$, then $\NP = \coNP$.
\end{problem}

\begin{solution}

  From the homework last week, we have proved that a problem $A \in \P \iff \overline{A} \in \P$.
  From the definition, we have $\coNP = \{\overline{A} \mid A \in \NP\}$,
  so if $\P = \NP$, then $\coNP = \{\overline{A} \mid A \in \P\}$.
  And this results in $\coNP = \P = \NP$.

\end{solution}

\begin{problem}
  For every $\TWOSAT$ instance $\varphi$ of $n$ variables, define graph
  $G_{\varphi}$ of $2n$ vertices as follows.
  For each variable $x_{i}$ in $\varphi$, $G_{\varphi}$ has two vertices labeled
  by $x_{i}$ and $\neg x_{i}$ respectively.
  There is a directed edge $\ell_{i} \to \ell_{j}$ if
  $(\neg \ell_{i}) \lor \ell_{j}$ or $\ell_{j} \lor (\neg \ell_{i})$ is a clause
  of $\varphi$.
  For notational convenience, for literal $\ell_{i} = \neg x_{k_{i}}$,
  $\neg \ell_{i}$ is defined to be $x_{k_{i}}$.
  Prove that $\varphi$ is unsatisfiable if and only if there exist
  paths from $x_{j}$ to $\neg x_{j}$ and from $\neg x_{j}$ to $x_{j}$ in
  $G_{\varphi}$ for some $j$.
  Use the above fact to show that $\TWOSAT \in \P$.
\end{problem}

\begin{solution}

  Suppose $\varphi$ is unsatisfiable,
  then for every assignment of variables,
  there must be two chains of clauses that
  coerce the assignment of $x_j$ to be true and false respectively.
  For example,
  \begin{equation*}
    (\lnot x_j) \lor l_{k_1}, (\lnot l_{k_1}) \lor l_{k_2}, \cdots, (\lnot l_{k_{m-1}}) \lor (\lnot x_j),
  \end{equation*}
  and
  \begin{equation*}
    x_j \lor l_{p_1}, (\lnot l_{p_1}) \lor l_{p_2}, \cdots, (\lnot l_{p_{t-1}}) \lor x_j,
  \end{equation*}
  where $l_{k_i}$ and $l_{p_i}$ are variables $x_t$ or its negation $\lnot x_t$.
  The former chain ensures that $x_j = \text{false}$,
  and the latter chain ensures that $x_j = \text{true}$,
  which leads to a contradiction so that $\varphi$ is unsatisfiable.
  From the first chain, we can see that there must be an edge from $x_j$ to $l_{k_1}$,
  from $l_{k_1}$ to $l_{k_2}$, $\cdots$, from $l_{k_{m-1}}$ to $\lnot x_j$,
  forming a path from $x_j$ to $\lnot x_j$.
  And from the second chain, we can see that there must be an edge from $\lnot x_j$ to $l_{p_1}$,
  from $l_{p_1}$ to $l_{p_2}$, $\cdots$, from $l_{p_{t-1}}$ to $x_j$,
  forming a path from $\lnot x_j$ to $x_j$.

  Suppose there exist paths from $x_j$ to $\lnot x_j$ and from $\lnot x_j$ to $x_j$ in $G_{\varphi}$.
  Note that the path from $x_j$ to $\lnot x_j$ ensures that $x_j = \text{false}$,
  because of the following chain:
  \begin{equation*}
    (\lnot x_j) \lor l_{k_1}, (\lnot l_{k_1}) \lor l_{k_2}, \cdots, (\lnot l_{k_{m-1}}) \lor (\lnot x_j),
  \end{equation*}
  where $l_{k_i}$ is a variable $x_t$ or its negation $\lnot x_t$.
  Similarly, the path from $\lnot x_j$ to $x_j$ ensures that $x_j = \text{true}$,
  which is a contradiction.
  Then $\varphi$ is unsatisfiable.

  Therefore, $\varphi$ is unsatisfiable if and only if there exist paths from $x_{j}$ to $\neg x_{j}$ and from $\neg x_{j}$ to $x_{j}$ in $G_{\varphi}$ for some $j$.
  It is easy to see that the existence of such paths can be checked in polynomial time,
  for we can use breadth-first search, starting from a vertex $x_j$,
  to find whether there is a path from $x_j$ to $\lnot x_j$ and vice versa.
  Thus, $\TWOSAT \in \P$.

\end{solution}

\begin{problem}
  The Lehmer's theorem states that a natural number $n$ is a prime number if and
  only if the following two conditions hold:
  \begin{enumerate}
    \item There is number $a$ such that $a^{n-1} \equiv 1 \pmod{n}$.
    \item For every prime factor $q$ of $n-1$, $a^{(n-1)/q} \not\equiv 1 \pmod{n}$.
  \end{enumerate}
  Use this theorem to show that $\PRIME \in \NP \cap \coNP$.
  (Hint: To prove $\PRIME \in \NP$, you may need to use recursively defined
  witness.)
\end{problem}

\begin{solution}

  We first prove that $\PRIME \in \NP$.
  Given a number $a$ and the prime factorization of $n-1$,
  we can verify in polynomial time:
  \begin{enumerate}
    \item $a^{n - 1} \equiv 1 \pmod{n}$.
      This is because we can calculate $a^{n-1} \mod n$ in $O(\log n)$ time
      by \href{https://en.wikipedia.org/wiki/Exponentiation_by_squaring}{exponentiation by squaring}.
    \item for every prime factor $q$ of $n-1$, $a^{(n-1)/q} \not\equiv 1 \pmod{n}$.
      This is because we can calculate $a^{(n-1)/q} \mod n$ in $O(\log n)$ time
      by \href{https://en.wikipedia.org/wiki/Exponentiation_by_squaring}{exponentiation by squaring},
      Also, there are only $O(\sqrt{n})$ such factor $q$ for $n - 1$,
      because factors appear in pairs.
      If one of the factor is greater than $\sqrt{n - 1}$,
      then the other one must be less than $\sqrt{n - 1}$.
      And an upper bound of the number of factors that $n - 1$ have is $O(\sqrt{n})$,
      because $n - 1$ can only have $\sqrt{n - 1}$ factors that are the `smaller ones.'
  \end{enumerate}
  Therefore, $\PRIME \in \NP$.

  Then we prove that $\PRIME \in \coNP$. Suppose $n$ is not a prime number
  and given $k$ where $k \mid n$, then we can verify this in polynomial time.
  Therefore, $\PRIME \in \coNP$.

  In conclusion, $\PRIME \in \NP \cap \coNP$.

\end{solution}

\end{document}
