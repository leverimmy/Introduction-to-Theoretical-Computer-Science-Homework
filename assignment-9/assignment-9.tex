\documentclass{homework}
\usepackage{bussproofs}
\EnableBpAbbreviations
\usepackage{ctex,hyperref,float,algorithm,algorithmic}

\name{熊泽恩} % Replace (Student Name) with your name.
\id{2022011223}
\term{2024 Spring}
\course{Introduction to Theoretical Computer Science}
\hwnum{9}

%\hwname{(Name)}          % Uncomment and replace (Name) with the type of the
                          % homework (e.g, Assignment, Problem Set, etc.) if you
                          % don't want the document to be labeled as "Homework."
%\problemname{(Name)}     % Uncomment and replace (Name) with the desired label
                          % for problems created with the problem environment.
%\solutionname{(Name)}    % Uncomment and replace (Name) with the desired label
                          % for solutions created with the solution environment.

% Load any other packages you need here.

\begin{document}

% \begin{problem}
%   Prove that any $\PSPACE$-hard problem is also $\NP$-hard.
% \end{problem}

% \begin{solution}

% \end{solution}

\begin{problem}
  Prove that $\TQBF$ restricted to formulas where the part following the
  quantifiers is in conjunctive normal form (CNF) is still $\PSPACE$-complete.
\end{problem}

\begin{lemma}
  (Tseytin Transformation)
  An example of the Tseytin Transformation is as follows.
  Denote $n$ as the length of $\psi$.
  Let $\psi = ((p \lor q) \land r) \to (\lnot s)$, then

  \begin{enumerate}
    \item Take all the subformulas of $\psi$:
          \begin{align*}
            &\lnot s, \\
            &p \lor q, \\
            &(p \lor q) \land r, \\
            &((p \lor q) \land r) \to (\lnot s).
          \end{align*}
          Since the number of subformulas equals to the number of operators,
          so there the number of subformulas is $O(n)$.
          \begin{align*}
            & x_1 \leftrightarrow \lnot s, \\
            & x_2 \leftrightarrow p \lor q, \\
            & x_3 \leftrightarrow x_2 \land r, \\
            & x_4 \leftrightarrow x_3 \to x_1.
          \end{align*}
          As a result of step(1), the number of variables which we have
          introduced is $O(n)$.
    \item Conjunct all substitutions and the substitution for $\psi$:
          $\psi' = x_4 \land (x_4 \leftrightarrow x_3 \to x_1) \land (x_3 \leftrightarrow x_2 \land r) \land (x_2 \leftrightarrow p \lor q) \land (x_1 \leftrightarrow \lnot s)$
    \item All substitutions can be transformed into CNF.
          Note that each subterm, for instance, $x_2 \leftrightarrow p \lor q$,
          is a 3-CNF. So expand them and we will gain at most $2^3 n$ formulas,
          which is still $O(n)$.
  \end{enumerate}
  Therefore, the length of $\psi'$ is linear to that of $\psi$.
  And it is clear that the Tseytin Transformation is polynomial-time computable.
\end{lemma}

\begin{solution}

  Denote the problem as $\TQBF(\text{CNF})$. Consider Turing machine $S$:

  \begin{enumerate}
    \item Suppose the input is $\tuple{\varphi}$. If $\varphi$ has no quantifiers,
          Then $\varphi$ has no variables. It either ``True'' or ``False.''
          Output accordingly.
    \item If $\varphi = \exists x\; \psi$, then evaluate $\psi$ with $x$
          equals true or false recursively. Accept if either case accepts.
          Reject otherwise.
    \item If $\varphi = \forall x\; \psi$, then evaluate $\psi$ with $x$
          equals true or false recursively. Accept if both case accepts.
          Reject otherwise.
  \end{enumerate}
  Each recursive level uses constant extra space, so that the space complexity
  of $S$ is $O(n)$. So $\TQBF(\text{CNF}) \in \PSPACE$.

  Now we will prove that $\TQBF(\text{CNF})$ is $\PSPACE$-complete by
  reducing $\TQBF$ to $\TQBF(\text{CNF})$.
  Given $\varphi = Q_1x_1Q_2x_2\cdots Q_kx_k\;\psi \in \TQBF$,
  then using the Tseytin Transformation,
  we can transform $\psi$ into $\psi'$ in polynomial time,
  and the length of $\psi'$ is linear to that of $\psi$.

  Therefore, $\TQBF(\text{CNF})$ is $\PSPACE$-complete.

\end{solution}

\begin{problem}
  Let $\problemsty{SUM} = \{ \tuple{x, y, z} \mid x, y, z > 0 \text{ are binary
    integers satisfying } x+y=z \}$.
  Show that $\problemsty{SUM} \in \L$.
\end{problem}

\begin{solution}

  It is not possible to calculate $x + y$ directly because we only can use
  an additional $O(\log n)$ space.
  However, we can construct an algorithm as follows.
  We use the pair $\tuple{p, c}$ to describe the current state,
  where $p$ is the position of the bit we're calculating,
  and $c \in \{0, 1\}$ is the carry of the result that we have already calculated.

  We can calculate the result from the lowest bit to the highest,
  bit by bit, increment $p$ by one after every calculation of each bit.

  Since $p$ will not exceed the length of input $n$,
  so $p \le n$, and the space complexity for $p$ is $O(\log n)$.
  Additionally, $c$ only needs $O(1)$ space to store.
  Therefore, $\problemsty{SUM} \in \L$.

\end{solution}

\begin{problem}
  \begin{parts}
    \part\label{a} An undirected graph is \emph{bipartite} if its nodes may be divided
    into two sets so that all edges go from a node in one set to a node in the
    other set.
    Show that a graph is bipartite if and only if it doesn't contain a cycle
    that has an odd number of nodes.
    \part\label{b} Let $\problemsty{BIPARTITE} = \{ \tuple{G} \mid G \text{ is a
      bipartite graph} \}$.
    Prove that $\problemsty{BIPARTITE}$ is in $\NL$.
  \end{parts}
\end{problem}

\begin{solution}

  \ref{a}

  Suppose that $G$ is bipartite, and let $V_1$ and $V_2$ be the two sets of
  nodes such that all edges go between $V_1$ to $V_2$. Each edge permits us
  to travel from a node in $V_1$ to a node in $V_2$ and vice versa. Thus, if
  we start at a node in $V_1$ and follow an edge, we must arrive at a node in
  $V_2$. If we follow another edge, we must arrive at a node in $V_1$. We can
  continue this process. If it forms a cycle, then the number of nodes in the
  cycle must be even since we alternate between nodes in $V_1$ and $V_2$.

  Suppose $G$ doesn't contain a cycle with an odd number of nodes. We can
  start at any node and assign it to $V_1$. Then, we assign all neighbors of
  nodes in $V_1$ to $V_2$, and all neighbors of nodes in $V_2$ to $V_1$. We
  continue this process until no more nodes can be assigned. Since all cycles
  have an even number of nodes, all nodes in the cycle will alternate
  between $V_1$ and $V_2$, and we won't have any conflicts. Thus, $G$ is
  bipartite.

  Therefore, a graph is bipartite if and only if it doesn't contain a cycle
  that has an odd number of nodes.

  \ref{b}

  Suppose $G = \tuple{V, E}$.
  Construct a nondeterministic Turing machine $M$ as follows.
  Denote the current state as $\tuple{v, l}$,
  where $v$ is the current vertex, $l$ is the length of the current path.
  For each vertex $v_{\text{init}} \in V$:
  \begin{enumerate}
    \item Start from $v_{\text{init}}$, and
    $\tuple{v, l} \gets \tuple{v_{\text{init}, 1}}$.
    \item 
      \begin{enumerate}
        \item Nondeterministically guess the next vertex $u$.
        \item If edge $(v, u) \in E$:
          \begin{itemize}
            \item If $u = v_{\text{init}}$,
            \begin{itemize}
              \item If $l$ is odd, return $\tuple{G} \notin \problemsty{BIPARTITE}$.
              \item If $l$ is even, return $\tuple{G} \in \problemsty{BIPARTITE}$.
            \end{itemize}
            \item Otherwise, $v \gets u, l \gets l + 1$.
          \end{itemize}
        \item If $l \ge |V|$, guess another vertex $u$.
      \end{enumerate}
    \item Return $\tuple{G} \in \problemsty{BIPARTITE}$.
  \end{enumerate}

  Since the length of the longest simple cycle in graph $G$ is less than $|V|$,
  so $l \le |V|$.
  Therefore, the cost of space complexity for storing $\tuple{v, l}$ is $O(\log n)$.

  Hence, $\problemsty{BIPARTITE}$ is in $\NL$.

\end{solution}

\begin{problem}
  Let $S(n) \ge \log n$ be a space-constructible function.
  Show that $\NSPACE(S(n)) = \class{coNSPACE}(S(n))$ is a consequence of
  $\NL = \class{coNL}$.
\end{problem}

\begin{solution}

  Suppose that $\NL = \class{coNL}$. Consider a problem $Q \in \NSPACE(S(n))$
  with an alphabet $\Sigma$.
  Since $S(n) \ge \log n$, so there exists $m \in \mathbb{N}$
  such that $S(n) = \log m$.
  Pad the input $\tuple{I}$ to $\tuple{I'} = \tuple{I\;c^{m - n}}$,
  where character $c \notin \Sigma$, so $|\tuple{I'}| = m$.

  Suppose the corresponding nondeterministic Turing machine for $Q$ is $M$.
  Construct a new nondeterministic Turing machine $M'$ as follows:

  \begin{enumerate}
    \item Parse the input as $\tuple{I'} = \tuple{I\;c^k}$.
    \item Use $M$ to solve $\tuple{I}$. Return the result.
  \end{enumerate}

  We have constructed a new nondeterministic Turing machine $M'$ that
  solves the problem $Q'$, where $Q'$ corresponds to $M'$ and $Q' \in \class{coNL}$.
  Now we will construct a new machine $M''$ that
  solves $Q$ and belongs to $\class{coNSPACE}(S(n))$.

  The machine $M''$ operates as follows:

  \begin{enumerate}
    \item On input $\tuple{I}$, where $|\tuple{I}| = n$:
    \item Pad the input to $\tuple{I'} = \tuple{I\;c^{m-n}}$, as before.
    \item Run $M'$ on input $\tuple{I'}$ to obtain the result $R$.
    \item If $R$ is "accept," then accept; otherwise, reject.
  \end{enumerate}

  The machine $M''$ simulates $M'$ on the padded input $\tuple{I'}$
  and returns the same result. Since $M'$ uses $\log m = S(n)$ space,
  $M''$ also uses $\log m = S(n)$ space.
  Therefore, $M''$ solves the problem $Q$ using $\log m = S(n)$ space,
  which means $Q \in \class{coNSPACE}(S(n))$. This leads to
  $\NSPACE(S(n)) \subseteq \class{coNSPACE}(S(n))$. Reverse the
  procedure above, then similarly, we can prove thate
  $\class{coNSPACE}(S(n)) \subseteq \NSPACE(S(n))$.

  Hence $\NSPACE(S(n)) = \class{coNSPACE}(S(n))$.

\end{solution}

\end{document}
