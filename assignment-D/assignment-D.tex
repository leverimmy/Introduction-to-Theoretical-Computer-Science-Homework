\documentclass{homework}
\usepackage{ctex,hyperref,float,algorithm,algorithmic,pgfplots}

\name{熊泽恩} % Replace (Student Name) with your name.
\id{2022011223}
\term{2024 Spring}
\course{Introduction to Theoretical Computer Science}
\hwnum{13}

%\hwname{(Name)}          % Uncomment and replace (Name) with the type of the
                          % homework (e.g, Assignment, Problem Set, etc.) if you
                          % don't want the document to be labeled as "Homework."
%\problemname{(Name)}     % Uncomment and replace (Name) with the desired label
                          % for problems created with the problem environment.
%\solutionname{(Name)}    % Uncomment and replace (Name) with the desired label
                          % for solutions created with the solution environment.

% Load any other packages you need here.

\begin{document}

\begin{problem}
  Prove that for every $\AM$ protocol for a language $A$, if Merlin and Arthur
  repeat the protocol $k$ times in parallel (Arthur runs $k$ independent random
  strings for each message and accepts only if all $k$ copies accept), then the
  probability that Arthur accepts $x \not\in A$ is at most $1/2^{k}$.
  (Recall that an $\AM$ protocol starts with Arthur sending the random string
  and Merlin replying a witness.
  You should not assume that the Merlin message for parallelized protocol is
  independent for each copy in your proof.)
\end{problem}

\begin{solution}

  Since all $k$ messages sent from Merlin to Arthur are parallelized,
  we can consider the $k$ messages as a single message $m$.
  Then the parallelized protocol is equivalent to the original $\AM$ protocol,
  where Arthur sends a random string and Merlin replies with $k$ witnesses.

  Let $E_i$ be the event that Arthur accepts $x \not\in A$ in the $i$-th copy.
  Formally, $E_i = (\exists P_0^*)\Pr(\pair{P^*}{V}(x) = 1) > 1/2$.
  Then the probability that Arthur accepts $x$ in the parallelized protocol is
  \begin{align*}
    \Pr(\text{Arthur accepts } x ) & = \Pr(\bigcap_{i=1}^{k} E_i) \\
    & = \Pr(E_1) \cdot \Pr(E_2 \mid E_1) \cdots \Pr(E_k \mid \bigcap_{i=1}^{k-1} E_i).
  \end{align*}

  Though the Merlin messages are not independent, we can still bound the probability
  for the $j$-th term $\Pr(E_j\mid \cap_{i = 1}^{j - 1}E_i)$. Due to the fact that
  Merlin would not receive any reply after Arthur first sent the random string,
  the $j$-th term is at most $1/2$. Therefore, the probability that Arthur accepts $x \not\in A$ in the parallelized
  protocol is at most $1/2^k$.

\end{solution}

\begin{problem}
  \begin{parts}
    \part\label{2.a} Explain why the following simulator does not work in establishing
    the zero-knowledge property of the protocol for $\GI$ discussed in the class.
    \begin{algorithmic}[1]
      \STATE{Choose $a\in \{0,1\}$ uniformly at random.}
      \STATE{Sample a random permutation $\pi$ and compute $G = \pi(G_a)$.}
      \STATE{Randomly sample $b \in \{0,1\}$.}
      \STATE{If $b = a$, output the transcript.
        Otherwise, rewind and start from the beginning.}
    \end{algorithmic}
    \part\label{2.b} Prove the zero-knowledge property of the protocol for $\GI$ discussed
    in the class formally.
  \end{parts}
\end{problem}

\begin{solution}

  \ref{2.a}
  The simulator does not work because it does not simulate the verifier's
  random tape. The verifier's random tape is used to generate the challenge
  $b$ in the protocol. This simulator doesn't even use the verifier to
  determine its output!   

  It is a plain guess-and-check simulator that
  outputs the transcript if the guess is correct, independent of
  that $G$ is isomorphic with $G_b$.
  The transcript does not have the same distribution as the real
  interaction between the verifier and the prover.
  This is not zero-knowledge because the simulator
  does not simulate the verifier's behavior at all.

  \ref{2.b}
  Consider a simulator $S$ that works as follows:
  \begin{algorithmic}[1]
    \STATE{Choose $a\in \{0,1\}$ uniformly at random.}
    \STATE{Sample a random permutation $\pi$ and compute $G = \pi(G_a)$.}
    \STATE{Randomly sample $r$ and simulate $V^*$ with $r$ as the random tape.}
    \STATE{If $V^*$ sends $b = a$, output $(G, \pi)$ as the message and the
            random tape $r$ as the internal randomness.}
    \STATE{If $V^*$ sends $b \neq a$, rewind and start from the beginning.}
  \end{algorithmic}

  We need to show that the output of the simulator $S$ is indistinguishable from
  the real interaction between the verifier $V^*$ and the prover $P$.
  It is clear that the output of $S$ is identically distributed to the real
  interaction because $a$ is chosen uniformly at random and $\pi$ is a random
  permutation. The only difference is that $S$ simulates the verifier's behavior
  with a random tape $r$, which will make no difference as $V^*$'s tape is
  also chosen uniformly at random.

  In addition, $S$ runs in expected polynomial time since the probability that it
  needs to rewind is $1/2$. This is because the probability that $b \neq a$
  is $1/2$ due to the fact that $a$ is chosen uniformly at random,
  independent of $V^*$.

\end{solution}

\end{document}
